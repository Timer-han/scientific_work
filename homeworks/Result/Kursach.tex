\documentclass[a4paper,12pt]{article}
\usepackage[utf8]{inputenc}
\usepackage[T2A]{fontenc}
\usepackage[russian]{babel}
\usepackage{amsmath,amssymb}
\usepackage{graphicx}
\usepackage{geometry}
\usepackage{cite}
\geometry{left=25mm,right=25mm,top=25mm,bottom=25mm}

\begin{document}

% ----------------------- Титульный лист -----------------------
\begin{titlepage}
  \centering
  {\large Московский государственный университет имени М.\,В.\,Ломоносова\\
  Механико-математический факультет\\
  Кафедра вычислительной математики}\\[2cm]

  {\LARGE Курсовая работа}\\[1cm]
  {\Large \textbf{«Численное моделирование фильтрации в пористой \\среде
  методом конечных объёмов»}}\\[1.5cm]

  {\large Выполнил: студент Петрунников Тимур Максимович\\
  Научный руководитель: Ануприенко Денис Валерьевич}\\[2cm]

  Москва, 2025
\end{titlepage}

\tableofcontents
\newpage

% ----------------------- Введение -----------------------
\section{Введение}
Фильтрация жидкости в пористой среде является ключевым процессом в гидрогеологии и инженерной практике. Течение подземных вод управляется законом Дарси и законом сохранения массы, что приводит к диффузионным уравнениям с нелинейными характеристиками среды. Практические задачи требуют учитывать сложную геометрию домена и неоднородность проницаемости, что делает аналитические решения невозможными. Метод конечных объёмов обеспечивает локальное сохранение массы и адаптивность к произвольным сеткам, поэтому он получил широкое применение в моделировании фильтрации.

Цель данной работы — разработка и исследование неявной конечнообъёмной схемы для моделирования однофазной фильтрации в пористой среде. Задачи:
\begin{itemize}
  \item провести обзор методов конечных разностей и конечных объёмов;
  \item сформулировать математическую модель стационарных и нестационарных задач фильтрации, включая уравнение Ричардса;
  \item реализовать неявную FVM-схему с методом простой итерации (Пикара);
  \item протестировать алгоритм на задаче Celia и проанализировать сходимость и водный баланс.
\end{itemize}
Результаты могут быть применены в экологическом мониторинге, проектировании ирригационных систем и моделировании миграции загрязнений.

% ----------------------- Обзор литературы -----------------------
\section{Обзор литературы}
Метод конечных разностей (МКР) предлагает простую аппроксимацию производных с использованием сеточных разностей и хорошо подходит для регулярных сеток \cite{Samarskii1989}. Однако для задач с сложной геометрией и сохранением массы МКР требует значительных доработок. 

Метод конечных объёмов (МКO) изначально развивался в гидрогазодинамике и теплообмене \cite{Bear1972}. Он базируется на интегральной формулировке законов сохранения в каждом контрольном объёме, что гарантирует строгий баланс массы. МКО применяют к произвольным сеткам и сложным доменам, сохраняя физическую консервативность \cite{Leontiev2017}.

В задачах ненасыщенной фильтрации уравнение Ричардса используется с зависимостями $\theta(h)$ и относительной проницаемостью $K_r(h)$ по моделям Ван-Генухтена–Муалема \cite{VanGenuchten1980,Celia1990}. Итерационные методы (Пикара, Ньютона) обеспечивают решение нелинейных систем с контролем невязки и массосбережения.

Таким образом, МКО сочетает в себе физическое обоснование, гибкость сетки и строгий закон сохранения, что делает его предпочтительным для моделирования фильтрации.

% ----------------------- Постановка задачи -----------------------
\section{Постановка задачи}
Рассматривается одномерная область $x\in[0,L]$ с пористой средой. Модель непрерывности и закон Дарси приводят к уравнению:
\begin{equation}
\phi\,\frac{\partial \theta(h)}{\partial t}
-
\frac{\partial}{\partial x}
\Bigl(K(h)\,\frac{\partial h}{\partial x}\Bigr)
= f(x,t),
\end{equation}
где $\phi$ — пористость, $\theta(h)$ — объёмное влагосодержание, $K(h)=K_s\,K_r(h)$ — проводимость, $f$ — источник/сток. Начальные условия:
\[
h(x,0) = h_{\mathrm{init}}(x),
\]
граничные условия:
\[
h(L,t)=H_{\mathrm{left}},\quad
\left.-K(h)\,\frac{\partial h}{\partial x}\right|_{x=0}=0.
\]

% ----------------------- Методика численного решения -----------------------
\section{Методика численного решения}
\subsection{Дискретизация по пространству и времени}
Разобьём область на $m$ ячеек длиной $\Delta x = L/m$. Объём $i$-й ячейки — $\Delta x$. Временной слой имеет шаг $\Delta t$. Используем неявную схему:
\[
\phi\,\Delta x\,
\frac{\theta_i^{n+1}-\theta_i^n}{\Delta t}
+
\Delta t\bigl(q_{i+\frac12}^{n+1}-q_{i-\frac12}^{n+1}\bigr)
= 0,
\]
\[
q_{i+\frac12}^{n+1}
= -K_{i+\frac12}
\frac{h_{i+1}^{n+1}-h_i^{n+1}}{\Delta x}.
\]

\subsection{Формулы для внутренних ячеек}
Для внутренних ячеек имеем:
\begin{align}
y_i\,\Delta x 
&\sim f\bigl(x_{i+\tfrac12}\bigr)\,\Delta x
= q_{i+1}^{n+1}-q_i^{n+1}
+ s_i\frac{h_i^{n+1}-h_i^n}{\Delta t}\,\Delta x
\nonumber\\
&= \frac{2K_i}{\Delta x}
\Bigl(
- h_{i-1}^{n+1}\frac{K_{i-1}}{K_i+K_{i-1}}
- h_{i+1}^{n+1}\frac{K_{i+1}}{K_i+K_{i+1}}
\nonumber\\
&\quad
+ h_i^{n+1}\frac{K_{i+1}(K_i+K_{i-1}) + K_{i-1}(K_i+K_{i+1})}
{(K_i+K_{i+1})(K_i+K_{i-1})}
\Bigr)
+ s_i\frac{h_i^{n+1}-h_i^n}{\Delta t}\,\Delta x
\nonumber\\
&\Rightarrow 
- h_{i-1}^{n+1}\frac{K_{i-1}}{K_i+K_{i-1}}
- h_{i+1}^{n+1}\frac{K_{i+1}}{K_i+K_{i+1}}
\nonumber\\
&\quad
+ h_i^{n+1}\Bigl(\frac{K_{i+1}(K_i+K_{i-1}) + K_{i-1}(K_i+K_{i+1})}
{(K_i+K_{i+1})(K_i+K_{i-1})}
+ s_i\frac{\Delta x^2}{2K_i\,\Delta t}\Bigr)
\nonumber\\
&= f\bigl(x_{i+\tfrac12}\bigr)\,\frac{\Delta x^2}{2K_i}
+ s_i\frac{\Delta x^2}{2K_i\,\Delta t}\,h_i^n.
\end{align}

\subsection{Левая ячейка}
\begin{align}
y_0\,\Delta x
&\sim f\bigl(x_{\tfrac12}\bigr)\,\Delta x
= -\frac{2K_1K_0}{K_0+K_1}\frac{h_1^{n+1}-h_0^{n+1}}{\Delta x}
+ \frac{2K_{-\tfrac12}K_0}{K_0+K_{-\tfrac12}}\frac{h_0^{n+1}-H_0}{\Delta x/2}
\nonumber\\
&\quad
+ s_0\frac{h_0^{n+1}-h_0^n}{\Delta t}\,\Delta x
\nonumber\\
&\Rightarrow
- \frac{K_1}{K_0+K_1}h_1^{n+1}
+ \Bigl(
\frac{2K_{-\tfrac12}(K_0+K_1)+K_1(K_0+K_{-\tfrac12})}
{(K_0+K_{-\tfrac12})(K_0+K_1)}
+ s_0\frac{\Delta x^2}{2K_0\,\Delta t}
\Bigr)\!h_0^{n+1}
\nonumber\\
&= \frac{\Delta x^2}{2K_0}f\bigl(x_{\tfrac12}\bigr)
+ \frac{2H_0}{K_0+K_{-\tfrac12}}
+ s_0\frac{\Delta x^2}{2K_0\,\Delta t}\,h_0^n.
\end{align}

\subsection{Правая ячейка}
\begin{align}
y_{m-1}\,\Delta x
&\sim f\bigl(x_{m-\tfrac12}\bigr)\,\Delta x
\nonumber\\
&=
- \frac{2K_{m-1}K_{m-\tfrac12}}{K_{m-1}+K_{m-\tfrac12}}
\frac{b-h_{m-1}}{\Delta x/2}
+ \frac{2K_{m-1}K_{m-2}}{K_{m-1}+K_{m-2}}
\frac{h_{m-1}-h_{m-2}}{\Delta x}
\nonumber\\
&\quad+ s_{m-1}\frac{h_{m-1}^{n+1}-h_{m-1}^n}{\Delta t}\,\Delta x
\nonumber\\
&\Rightarrow
\Bigl(
\frac{2K_{m-\tfrac12}(K_{m-1}+K_{m-2})+K_{m-2}(K_{m-1}+K_{m-\tfrac12})}
{(K_{m-1}+K_{m-\tfrac12})(K_{m-1}+K_{m-2})}
+ s_{m-1}\frac{\Delta x^2}{2K_{m-1}\,\Delta t}
\Bigr)\!h_{m-1}^{n+1}
\nonumber\\
&\quad
- \frac{K_{m-2}}{K_{m-1}+K_{m-2}}h_{m-2}^{n+1}
\nonumber\\
&= \frac{\Delta x^2}{2K_{m-1}}f\bigl(x_{m-\tfrac12}\bigr)
+ \frac{2b}{K_{m-1}+K_{m-\tfrac12}}
+ s_{m-1}\frac{\Delta x^2}{2K_{m-1}\,\Delta t}\,h_{m-1}^n.
\end{align}

% ----------------------- Программная реализация -----------------------
\section{Программная реализация}
Код реализован на Python с использованием NumPy и SciPy. Основной алгоритм:
\begin{enumerate}
  \item Создание сетки $x_i$, начальные условия $h_i^0$, параметры $\phi$, $K_s$, параметры модели Ван-Генухтена.
  \item Цикл по времени: для каждого шага $n\to n+1$:
    \begin{enumerate}
      \item Вычисление $\theta_i^n=\theta(h_i^n)$ по \eqref{eq:vangenuchten}.
      \item Эффективная проводимость $K_{i+\frac12} = \bigl(K_sK_r(h^n)\bigr)$.
      \item Сборка системы $A\,h^{n+1}=b$ с учётом дискретизации.
      \item Решение линейной системы (метод Томаса или \texttt{scipy.linalg.solve}).
      \item Обновление $h_i^{n+1}$ и $\theta_i^{n+1}$.
    \end{enumerate}
  \item Контроль водного баланса и невязки.
\end{enumerate}

% ----------------------- Тестовая задача Celia -----------------------
\section{Тестовая задача Celia}
Классический тест Целии \cite{Celia1990}: $L=0.4$\,м, начальные $h(x,0)$ заданы из референсных данных, на $x=L$ в $(t>0)$ поддерживается $h=0$, на $x=0$ поток $q=0$. Параметры $\phi=0.287$, $\theta_r=0.075$, $\alpha=1.611\times10^{6}$, $n=3.96$, $K_s=9.44\times10^{-5}\,\mathrm{м/с}$. Результаты моделирования сравниваются с референсными профилями и проверяется сохранение массы.

% ----------------------- Результаты и обсуждение -----------------------
\section{Результаты и обсуждение}
\begin{figure}[ht]
  \centering
  \includegraphics[width=0.8\textwidth]{front_filtration.png}
  \caption{Фронт фильтрации на разных временных срезах.}
  \label{fig:front}
\end{figure}

\begin{figure}[ht]
  \centering
  \includegraphics[width=0.8\textwidth]{residual.png}
  \caption{Невязка решения по итерациям метода Ньютона.}
  \label{fig:residual}
\end{figure}

\begin{figure}[ht]
  \centering
  \includegraphics[width=0.8\textwidth]{water_balance.png}
  \caption{Водный баланс: накопленная масса vs поступление.}
  \label{fig:balance}
\end{figure}

Сравнение профилей показывает согласие с данными Целии. Невязка быстро убывает до $\varepsilon\sim10^{-6}$. Водный баланс соблюдается в пределах численной погрешности.

% ----------------------- Заключение -----------------------
\section{Заключение}
Разработана и протестирована неявная конечнообъёмная схема для моделирования однофазной фильтрации в пористой среде. Алгоритм демонстрирует хорошую сходимость и строгую сохранность массы. Тест Целии подтвердил физическую корректность и точность решения. В последующих работах рекомендуется применять метод Ньютона с автодифференцированием для ускорения сходимости и расширять схему на многомерные задачи.

% ----------------------- Библиография -----------------------
\begin{thebibliography}{9}
\bibitem{Bear1972}
J. Bear. \emph{Dynamics of Fluids in Porous Media}. New York: Elsevier, 1972.

\bibitem{Celia1990}
M.\,A. Celia, E.\,T. Bouloutas, R.\,L. Zarba. A general mass-conservative numerical solution of Richards' equation. \emph{Water Resources Research}, 26(7):1483–1496, 1990.

\bibitem{VanGenuchten1980}
M.\,T. van Genuchten. Closed-form equation for predicting the hydraulic conductivity of unsaturated soils. \emph{Soil Sci. Soc. Am. J.}, 44(5):892–898, 1980.

\bibitem{Samarskii1989}
А.\,А. Самарский, А.\,Н. Гулин. \emph{Методы конечных разностей}. М.: Наука, 1989.

\bibitem{Leontiev2017}
Н.\,Е. Леонтьев. \emph{Основы теории фильтрации}. М.: МГУ, 2017.
\end{thebibliography}

% Чтобы отключить библиографию, закомментируйте блок \begin{thebibliography}...\end{thebibliography} 
% и удалите вызов пакета \usepackage{cite} при необходимости.

\end{document}
